\documentclass[a4paper,notitlepage]{article}
\usepackage{authblk}
\usepackage{graphicx}
\usepackage{hyperref}
\usepackage{siunitx}
\usepackage{url}
\begin{document}
\title{The theory of Light Emitting Diode (LED) reversing attacks}
\author{Joe Loughry}
\affil{University of Denver \\
Ritchie School of Engineering and Computer Science\footnote{Email
\href{mailto:joe.loughry@cs.du.edu}{joe.loughry@cs.du.edu}}}
\affil{and \href{https://netoir.com}{Netoir.com}}
\maketitle
\begin{abstract}
  LED reversing attacks depend on the relationship between LED photoelectric
  current and the logic threshold value of the connection GPIO.
\end{abstract}
\maketitle
\section{Introduction}
LED reversing attacks were introduced in an oral presentation at EMC Europe
2018 (Amsterdam) and first described in print in 2019 \cite{Loughry2019}.
\section{Theory of Operation}
Relationship between LED photocurrent, voltage, and the logic threshold of the
connection GPIO.
\subsection{Predictions}
\section{What Systems are Affected}
Arduino. Not the Raspberry Pi. Atmel ATSAMD21.
\subsection{Arduino}
Only the Arduino R1, b/c of the op-amp in R3.
\subsection{Voting Machines}
Example of what embedded systems might be at risk.
\bibliographystyle{plain}
\bibliography{consolidated_bibtex_file}
\end{document}

