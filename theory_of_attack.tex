\documentclass[a4paper,notitlepage]{article}
\usepackage{authblk}
\usepackage{graphicx}
\usepackage{hyperref}
\usepackage{minted}
\usepackage{siunitx}
\usepackage{textcomp}
\usepackage{url}
\begin{document}
\title{The theory of Light Emitting Diode (LED) reversing attacks}
\author{Joe Loughry}
\affil{University of Denver \\
Ritchie School of Engineering and Computer Science\footnote{Email
\href{mailto:joe.loughry@cs.du.edu}{joe.loughry@cs.du.edu}}}
\affil{and \href{https://netoir.com}{Netoir.com}}
\maketitle
\begin{abstract}
  LED reversing attacks depend on the relationship between LED photoelectric
  current and the logic threshold value of the connection GPIO.
\end{abstract}
\maketitle
\section{Introduction}
LED reversing attacks were introduced in an oral presentation at EMC Europe
2018 (Amsterdam) and first described in print in 2019 \cite{Loughry2019}.
\section{Theory of Operation}
Relationship between LED photocurrent, voltage, and the logic threshold of the
connection GPIO.
\section{Digital and Analogue GPIO pins}
This is only using digital pins. Analogue I/O pins, available on some
computers, open up a wider realm of possibility.
\subsection{Predictions}A
\begin{enumerate}
  \item LED photocurrent\textrightarrow voltage on GPIO pin\textrightarrow
    \SI{3.3}{\volt} or \SI{5}{\volt} logic threshold\textrightarrow
    vulnerability.
  \item The pin 13 LED on Arduino is active during uploading. The bootloader
    subsystem on Arduino might be directly vulnerable and would serve the
    purpose of the malware.
\end{enumerate}
\section{What Systems are Affected}
Arduino. Not the Raspberry Pi. Atmel ATSAMD21.
\subsection{Arduino}
Only the Arduino R1, b/c of the op-amp in R3.
\subsection{Microcontrollers and SoCs}
ATmega, ARM Cortex, Intel Quark, ATTiny at least.
\subsection{Voting Machines}
Example of what embedded systems might be at risk.
\bibliographystyle{plain}
\bibliography{consolidated_bibtex_file}
\appendix
\section{Arduino Sketches}
This was the first proof-of-concept to work correctly. It runs on an Adafruit
Trinket M0 (Atmel ATSAMD21). The red `L' LED flashes once a second (to make it
easier to aim the laser) and whenever it's dark, it's listening.
\begin{minted}{c}
// the setup function runs once when you press reset or power the board
void setup() {
  pinMode(LED_BUILTIN, OUTPUT);
  // Look in the Serial Monitor window for results.
  Serial.begin(9600);
}

// the loop function runs over and over again forever
void loop() {
  digitalWrite(LED_BUILTIN, HIGH);   // turn LED on
  delay(50);                         // 50 ms
  digitalWrite(LED_BUILTIN, LOW);    // turn the LED off
  // Following is the attack listening malware; it runs for 1 s.
  pinMode(LED_BUILTIN, INPUT);
  for (int i=0; i<5; i++) {
    int level = digitalRead(LED_BUILTIN);
    Serial.println(level);
    delay(200);
  }
  pinMode(LED_BUILTIN, OUTPUT);
}
\end{minted}
This is driving a couple of Adafruit 14-segment alphanumeric displays via
I\textsuperscript{2}C. It blinks the pin 13 LED and also continuously scrolls
`NORMAL OPERATION' on the alphanumeric display until it sees a start bit on the
pin 13 LED. When it does, it drops into the \texttt{loop()} function.
\begin{minted}{c}
#include <Wire.h>
#include <Adafruit_GFX.h>
#include "Adafruit_LEDBackpack.h"

Adafruit_AlphaNum4 alpha4 = Adafruit_AlphaNum4();

char string_to_display[] = "        NORMAL OPERATION";

char canary[] = "CANARY";

char displaybuffer[4] = {' ', ' ', ' ', ' '};
char largedisplaybuffer[8] = {' ', ' ', ' ', ' ', ' ', ' ', ' ', ' '};

void first_module(void);
void second_module(void);
void safe_alpha_write(int digit, int i);
void lamp_test(void);
void clear_display(void);

void setup() {
  bool valid_start_bit = false;

  Serial.begin(9600);
  Serial.println("In setup()");

  clear_display();

  for(;;) {
    int time_since_last_blink = 0; // milliseconds
    for (int i = 0; i < (int)strlen(string_to_display); i++) {
      int elapsed_time = 0; // milliseconds
      const int blink_interval = 1000; // milliseconds
      const int display_update_interval = 150; // milliseconds
      const int bit_sampling_time = 5; // milliseconds
      const int bit_interval = 1000; // milliseconds
      const int tolerance = 500; // milliseconds

      first_module();
      for (int position = 0; position < 4; position++) {
        safe_alpha_write(position, i + position);
      }
      alpha4.writeDisplay();

      second_module();
      for (int position = 0; position < 4; position++) {
        safe_alpha_write(position, i + position + 4);
      }
      alpha4.writeDisplay();

      // Only blink the LED if it's time to.
      if (time_since_last_blink >= blink_interval) {
        pinMode(LED_BUILTIN,OUTPUT);
        digitalWrite(LED_BUILTIN, HIGH);
        delay(50);
        digitalWrite(LED_BUILTIN, LOW);
        time_since_last_blink = 0;
      }

      pinMode(LED_BUILTIN, INPUT);

      // I'm not sure if this first digitalRead() is necessary.
      (void)digitalRead(LED_BUILTIN);

      while(digitalRead(LED_BUILTIN)) {
        delay(bit_sampling_time);
        time_since_last_blink += bit_sampling_time;
        elapsed_time += bit_sampling_time;
      }
      if (elapsed_time > 0) {
        Serial.println("I saw light.");
        if (elapsed_time < bit_interval - tolerance) {
          Serial.print("But it was too short; elapsed time = ");
          Serial.print(elapsed_time);
          Serial.println("ms.");
          continue;
        }
        else if (elapsed_time > bit_interval + tolerance) {
          Serial.print("But it was too long; elapsed_time = ");
          Serial.print(elapsed_time);
          Serial.println("ms.");
          continue;
        }
        else {
          Serial.print("I think I saw a start bit. It lasted ");
          Serial.print(elapsed_time);
          Serial.println(" milliseconds.");
          valid_start_bit = true;
          break;
        }
      }
      delay(display_update_interval - elapsed_time);
      time_since_last_blink += (display_update_interval - elapsed_time);
      elapsed_time = 0; // Reset the timer.
    }
    if (valid_start_bit) {
      break;
    }
  }
}

void loop() {
  Serial.println("I'm listening.");
  delay(10000);
}

void safe_alpha_write(int digit, int i) {
  if(i < (int)strlen(string_to_display)) {
        alpha4.writeDigitAscii(digit, string_to_display[i]);
  }
  else {
    alpha4.writeDigitAscii(digit, ' ');
  }
}

void lamp_test(void) {
  alpha4.begin(0x70);
  alpha4.writeDigitRaw(0, 0xFFFF);
  alpha4.writeDisplay();
  delay(50);
  alpha4.writeDigitRaw(0, 0x0);
  alpha4.writeDigitRaw(1, 0xFFFF);
  alpha4.writeDisplay();
  delay(50);
  alpha4.writeDigitRaw(1, 0x0);
  alpha4.writeDigitRaw(2, 0xFFFF);
  alpha4.writeDisplay();
  delay(50);
  alpha4.writeDigitRaw(2, 0x0);
  alpha4.writeDigitRaw(3, 0xFFFF);
  alpha4.writeDisplay();
  delay(50);
  alpha4.writeDigitRaw(3, 0x0);
  alpha4.writeDisplay();
  alpha4.begin(0x71);
  alpha4.writeDigitRaw(0, 0xFFFF);
  alpha4.writeDisplay();
  delay(50);
  alpha4.writeDigitRaw(0, 0x00);
  alpha4.writeDigitRaw(1, 0xFFFF);
  alpha4.writeDisplay();
  delay(50);
  alpha4.writeDigitRaw(1, 0x0);
  alpha4.writeDigitRaw(2, 0xFFFF);
  alpha4.writeDisplay();
  delay(50);
  alpha4.writeDigitRaw(2, 0x0);
  alpha4.writeDigitRaw(3, 0xFFFF);
  alpha4.writeDisplay();
  delay(50);
  alpha4.writeDigitRaw(3, 0x0);
  alpha4.writeDisplay();
}

void clear_display(void) {
  first_module();
  alpha4.clear();
  alpha4.writeDisplay();
  second_module();
  alpha4.clear();
  alpha4.writeDisplay();
}

void first_module(void) {
  alpha4.begin(0x70);
}

void second_module(void) {
  alpha4.begin(0x71);
}
\end{minted}
The receiver is working now:
\begin{minted}{c}
#include <Wire.h>
#include <Adafruit_GFX.h>
#include "Adafruit_LEDBackpack.h"

Adafruit_AlphaNum4 alpha4 = Adafruit_AlphaNum4();

char string_to_display[] = "        NORMAL OPERATION";

char canary[] = "CANARY";

char large_display_buffer[8] = {' ', ' ', ' ', ' ', ' ', ' ', ' ', ' '};

const int bit_interval = 1000; // milliseconds
const int tolerance = 500; // milliseconds
const int bit_sampling_time = 5; // milliseconds

void first_module(void);
void second_module(void);
void safe_alpha_write(int digit, int i);
void lamp_test(void);
void clear_display(void);

void setup() {
  bool valid_start_bit = false;

  Serial.begin(9600);
  for (int i=0; i<10; i++) {
    Serial.print("In setup(); (");
    Serial.print(i);
    Serial.println("/10");
  }

  clear_display();

  for(;;) {
    int time_since_last_blink = 0; // milliseconds
    for (int i = 0; i < (int)strlen(string_to_display); i++) {
      int elapsed_time = 0; // milliseconds
      const int blink_interval = 1000; // milliseconds
      const int display_update_interval = 150; // milliseconds

      first_module();
      for (int position = 0; position < 4; position++) {
        safe_alpha_write(position, i + position);
      }
      alpha4.writeDisplay();

      second_module();
      for (int position = 0; position < 4; position++) {
        safe_alpha_write(position, i + position + 4);
      }
      alpha4.writeDisplay();

      // Only blink the LED if it's time to.
      if (time_since_last_blink >= blink_interval) {
        pinMode(LED_BUILTIN, OUTPUT);
        digitalWrite(LED_BUILTIN, HIGH);
        delay(50);
        digitalWrite(LED_BUILTIN, LOW);
        time_since_last_blink = 0;
      }

      pinMode(LED_BUILTIN, INPUT);

      // I'm not sure if this first digitalRead() is necessary.
      (void)digitalRead(LED_BUILTIN);

      while(digitalRead(LED_BUILTIN)) {
        delay(bit_sampling_time);
        time_since_last_blink += bit_sampling_time;
        elapsed_time += bit_sampling_time;
      }
      if (elapsed_time > 0) {
        Serial.println("I saw light.");
        if (elapsed_time < bit_interval - tolerance) {
          Serial.print("But it was too short; elapsed time = ");
          Serial.print(elapsed_time);
          Serial.println("ms.");
          continue;
        }
        else if (elapsed_time > bit_interval + tolerance) {
          Serial.print("But it was too long; elapsed_time = ");
          Serial.print(elapsed_time);
          Serial.println("ms.");
          continue;
        }
        else {
          Serial.print("I think I saw a start bit. It lasted ");
          Serial.print(elapsed_time);
          Serial.println(" milliseconds.");
          valid_start_bit = true;
          break;
        }
      }
      delay(display_update_interval - elapsed_time);
      time_since_last_blink += (display_update_interval - elapsed_time);
      elapsed_time = 0; // Reset the timer.
    }
    if (valid_start_bit) {
      break;
    }
  }
}

int digit_position = 0;

void loop() {
  int bits[8] = {0, 0, 0, 0, 0, 0, 0, 0};
  int elapsed_time = 0; // milliseconds

  Serial.print("I'm listening.");

  // Assume bits are sent most significant bit first.
  for (int i=0; i<7; i++) {
    delay(bit_interval / 2);
    bits[i] = digitalRead(LED_BUILTIN);

    // Flash the LED very fast to say we got a bit.
    pinMode(LED_BUILTIN, OUTPUT);
    digitalWrite(LED_BUILTIN, HIGH);
    delay(25);
    digitalWrite(LED_BUILTIN, LOW);

    // Return LED to input state.
    pinMode(LED_BUILTIN, INPUT);
  }

  // Finish out the last bit interval.
  delay(bit_interval / 2);
  // Wait out the stop bit;
  delay(bit_interval);

  // Long flash to say we got a byte.
  pinMode(LED_BUILTIN, OUTPUT);
  digitalWrite(LED_BUILTIN, HIGH);
  delay(500);
  digitalWrite(LED_BUILTIN, LOW);

  // Decode the bits and display what we got.
  char byte = 0;

  for (int i=7; i>=0; i--) {
    byte += (bits[i] << i);
    // Clear out bits for next time.
    bits[i] = 0;
  }

  Serial.print("Got \"");
  Serial.print(byte);
  Serial.println("\".");

  Serial.print("digit_position = ");
  Serial.println(digit_position);

  large_display_buffer[digit_position++] = byte;
  if (digit_position > 7) {
    digit_position = 0;
  }

  first_module();
  for (int i = 0; i < 4; i++) {
    alpha4.writeDigitAscii(i, large_display_buffer[i]);
  }
  alpha4.writeDisplay();

  second_module();
  for (int i = 0; i < 4; i++) {
    alpha4.writeDigitAscii(i, large_display_buffer[4 + i]);
  }
  alpha4.writeDisplay();

  // Wait for the next start bit.
  pinMode(LED_BUILTIN, INPUT);
  while(!digitalRead(LED_BUILTIN)) {
    delay(bit_sampling_time);
  }
  while(digitalRead(LED_BUILTIN)) {
    delay(bit_sampling_time);
    elapsed_time += bit_sampling_time;
  }
  if (elapsed_time > 0) {
    Serial.println("I saw light.");
    if (elapsed_time < bit_interval - tolerance) {
      Serial.print("But it was too short; elapsed time = ");
      Serial.print(elapsed_time);
      Serial.println("ms.");
    }
    else if (elapsed_time > bit_interval + tolerance) {
      Serial.print("But it was too long; elapsed_time = ");
      Serial.print(elapsed_time);
      Serial.println("ms.");
    }
    else {
      Serial.print("I think I saw a start bit. It lasted ");
      Serial.print(elapsed_time);
      Serial.println(" milliseconds.");
    }
  }
}

void safe_alpha_write(int digit, int i) {
  if(i < (int)strlen(string_to_display)) {
        alpha4.writeDigitAscii(digit, string_to_display[i]);
  }
  else {
    alpha4.writeDigitAscii(digit, ' ');
  }
}

void lamp_test(void) {
  alpha4.begin(0x70);
  alpha4.writeDigitRaw(0, 0xFFFF);
  alpha4.writeDisplay();
  delay(50);
  alpha4.writeDigitRaw(0, 0x0);
  alpha4.writeDigitRaw(1, 0xFFFF);
  alpha4.writeDisplay();
  delay(50);
  alpha4.writeDigitRaw(1, 0x0);
  alpha4.writeDigitRaw(2, 0xFFFF);
  alpha4.writeDisplay();
  delay(50);
  alpha4.writeDigitRaw(2, 0x0);
  alpha4.writeDigitRaw(3, 0xFFFF);
  alpha4.writeDisplay();
  delay(50);
  alpha4.writeDigitRaw(3, 0x0);
  alpha4.writeDisplay();
  alpha4.begin(0x71);
  alpha4.writeDigitRaw(0, 0xFFFF);
  alpha4.writeDisplay();
  delay(50);
  alpha4.writeDigitRaw(0, 0x00);
  alpha4.writeDigitRaw(1, 0xFFFF);
  alpha4.writeDisplay();
  delay(50);
  alpha4.writeDigitRaw(1, 0x0);
  alpha4.writeDigitRaw(2, 0xFFFF);
  alpha4.writeDisplay();
  delay(50);
  alpha4.writeDigitRaw(2, 0x0);
  alpha4.writeDigitRaw(3, 0xFFFF);
  alpha4.writeDisplay();
  delay(50);
  alpha4.writeDigitRaw(3, 0x0);
  alpha4.writeDisplay();
}

void clear_display(void) {
  first_module();
  alpha4.clear();
  alpha4.writeDisplay();
  second_module();
  alpha4.clear();
  alpha4.writeDisplay();
}

void first_module(void) {
  alpha4.begin(0x70);
}

void second_module(void) {
  alpha4.begin(0x71);
}
\end{minted}
\end{document}

